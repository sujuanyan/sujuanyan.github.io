\documentclass[margin]{res}
\setlength{\paperwidth}{210mm}
\setlength{\paperheight}{300mm}

\setlength{\resumewidth}{180mm}
\setlength{\topmargin}{-12mm}

\setlength{\oddsidemargin}{-8mm}
\setlength{\textwidth}{140mm}
\setlength{\sectionwidth}{35mm}

\setlength{\textheight}{260mm}
\usepackage{setspace}
\usepackage{color}
\renewcommand{\rmdefault}{ppl}
\linespread{0.94}
\DeclareFixedFont{\titlefont}{OT1}{qbk}{m}{n}{20pt}
\usepackage{times}
\usepackage{url}
\usepackage{booktabs}

\usepackage[colorlinks=true]{hyperref}
\hypersetup{
     colorlinks = true,
     linkcolor = blue,
     anchorcolor = blue,
     citecolor = blue,
     filecolor = blue,
     urlcolor = blue
     }
%% new commands ----------------------------------------------------

%%---------------------------------------------------------------------------------------------
\begin{document}
\newcommand{\header}[1]{
\section{\textcolor[rgb]{0.02,0.54,0.82}{\vbox{\hrule width 90pt height 8pt}}}
			\textbf{\textcolor[rgb]{0.02, 0.54, 0.82} {\fontencoding{OT1}\fontfamily{cmr}\fontseries{m}\fontshape{n}\fontsize{24.88}{10}\selectfont \large{#1}}}
        }

\newcommand{\RISSpaper}{
    \textbf{J. Su.} 
        \\ Hardware  Acceleration for Sensor Data Fetch,
        \\ {Carnegie Mellon Robotics Institute Summer Scholars Working Papers Journal Vol.6, pp. 129-132, 2018} 
         \href{https://riss.ri.cmu.edu/wp-content/uploads/2018/11/RISS_Journal_Nov26-r.pdf}{[pdf]}
        }
\newcommand{\GPEpaper}{
    \textbf{J. Su}, Y. Zha, K. Wang, M.E. Villanueva, R. Paulen, B. Houska. 
        \\Interval Superposition Arithmetic for Guaranteed Parameter Estimation
        \\ Dynamics and Control of Process Systems, including Biosystems,2019 
         \href{http://arxiv.org/abs/1810.11967}{[pdf]}
        }

%%-----------------------------------------------------------------
% Name and basic info
{
     \moveleft 0.5\hoffset\centerline
     {
        \bf\textcolor[rgb]
        {0.02,0.54,0.82}
        {\fontencoding{OT1}\fontfamily{cmr}\fontseries{m}\fontshape{n}\fontsize{24.88}{10}\selectfont Junyan Su}
        \textcolor[rgb]        
        {0.4,0.4,0.4}
        {
            %\huge \textbardbl \hspace{2mm}
            %{\titlefont Curriculum Vitae}
        }
     }
    % Draw a horizontal line the whole width of resume:
     \moveleft1.08\hoffset\vbox{\hrule width \resumewidth height 1pt}
     \smallskip
    % Contact info  --------------------------------------------
    % Again, the address lines must be centered over entire width of resume:
     %\moveleft0.5\hoffset\centerline
     {
        %\textcolor[rgb]
        %{0.3,0.3,0.3}{0,0,0}
        %{ShanghaiTech University $\bullet$ 393,Huaxia Road,Shanghai, China, 201210}
     }
     \moveleft0.5\hoffset\centerline
    {
        \textcolor[rgb]
        {0,0,0}
        {
            %Mobile: (+86) 150-0212-7975 $\bullet$\\
             \textcolor{black}{\url{sujy@berkeley.edu}} /
            \url{sujy@shanghaitech.edu.cn}
        }     
    }
    \moveleft0.5\hoffset\centerline
    {
        \textcolor[rgb]
        {0,0,0}
        {
            \url{https://sujunyan.github.io/} 
        }     
    }
} %% basic info and name done
%% resume begin ------------------------------------------------------

    \begin{resume}
   %% EDUCATION begin ----------------------------------------------
    \header{EDUCATION}
    {
\section{\centerline{\textcolor[rgb]{0, 0, 0}{Sept.2015-Jun.2019}}}
            \textbf {ShanghaiTech University}
            \hspace{75mm}
            \textbf{Shanghai, China}
			\newline
			B.E. Candidate in Computer Science and Technology
            \newline
            GPA: \textbf{3.84 /4.0} 
            Ranking : \textbf{3/95}

           \vspace{-3mm}
 \section{\centerline{\textcolor[rgb]{0, 0, 0}{Aug.2018-May 2019}}}
            \textbf {University of California at Berkeley} 
            \hspace{70mm}
            \textbf {CA, USA}
            \newline
            Concurrent Enrollment Student at College of Engineering 

           \vspace{-3mm}
\section{\centerline{\textcolor[rgb]{0, 0, 0}{Aug.2019-present}}}
            \textbf {Washington University in St. Louis} 
            \hspace{70mm}
            \textbf {MO, USA}
            \newline
            Ph.D. student in Systems Science and Mathematics
        
} 

% EDUCATION done
    %% research interest ----------------------------------------
    \header{RESEARCH INTERESTS}
    {
        \\Control Theory  
        \\Optimal Control  
        \\Optimization
    }
    
    %% Publication begin -----------------------------------------
    \header{PUBLICATIONS}
    {
    \begin{itemize}
    \item \GPEpaper
    %\item \RISSpaper
    \end{itemize}
        
    }
    %% Publication end 
    
%% awards & honors ----------------------------------------
\header{HONORS \& AWARDS}
{
\section{\centerline{2016,2017}}
            Scholarship for Academic Excellence, ShanghaiTech University
            \vspace{-3mm}
            
\section{\centerline{Oct.13 2017}}
    Most Innovative Robot in Rescue Robot Competition, \\
    IEEE International Symposium on Safety, Security and Rescue Robotics
            \vspace{-3mm}
}   % awards & honors done
 

    
%% research experience begin --------------------------------------------
    \header{RESEARCH EXPERIENCE}
{
% RISS
{
\section{  \centerline{\textcolor[rgb]{0, 0, 0}{Jun.2018-Aug.2018}}}
           {\textbf { Carnegie Mellon University} }
           \hspace{65mm}
           \textbf{Pittsburgh, PA, USA}
            Robotics Institute Summer Scholars Program \\
            Advisors: Prof. Howie Choset \& Lu Li
           \newline
           To design one logic-circuit-level layout with Verilog to fetch data from multiple sensors and reduce CPU intervention time.
}           
% Robomasters
{
\section{\centerline{\textcolor[rgb]{0, 0, 0}{Sept.2017-May 2018}}}
           \textbf {Robomasters 2018}         
           \hspace{89mm}
           \textbf{Nanjing,China}
           \newline Advisor: Prof. Andre Rosendo \\
           \href{https://www.robomaster.com/en-US/robo/rm}{RoboMaster} is one international robotics competition.
           The competition is like multiplayer online battle arena (MOBA) video game.
           Each team will build their own robots that serve different 
           functionality. 
}
%interval model          
{
\section{\centerline{\textcolor[rgb]{0, 0, 0}{May 2018-Jan.2019}}}
           \textbf {A Software for Interval Superposition Model}         
           \hspace{49mm}
           \textbf{Shanghai,China}
           \newline Advisor: Prof. Boris Houska \\
           Our software package provides a tool to construct enclosures 
           of the image set of nonlinear functions easily and 
           efficiently, which is needed by a wide variety of numerical 
           computing and control algorithms. 
} % ISA done
}       % experience done
           
%% course projects begin --------------------------------
\header{COURSE PROJECTS}
{
    \begin{itemize}
        \item  Lego Pick \& Place Assembler \href{https://legopicknplace.wixsite.com/me102b}{[website]}.
        \item  Turtlebot with Robotic Arm Delivery 
        \href{https://sites.google.com/view/auto-search-pickup-vehicle}{[website]}.
        \item  A Don't-Touch-Me Robot \href{https://sujunyan.github.io/FABLAB_DOCUMENTATION/final_project/final_project.html}{[website]}
        \item Completed and passed all the points in the \href{http://www.scs.stanford.edu/18wi-cs140/pintos/pintos.html#SEC_Contents}{[Pintos project]}
        \item Optimal 800MHz 6-Bit “Absolute-value Detector”  \\
        In this project, I and my teammate implemented a CMOS level circuit “Absolute-value Detector” with Cadence Virtuoso. We achieved the minimum delay compared with other teams in the course.
        
    \end{itemize}
} %% course projects end


    %% technical skills ---------------------------------------
\header{TECHNICAL SKILLS}
{
\section{}
\begin{spacing}{1.0}
            \textbf{Programming Languages:} C/C++, Python\\
            \textbf{Scientific Tools:} MATLAB, Mathematica, Julia, ROS \\
            \textbf{Hardware Design:}pSoC, STM32xx,Verilog,Cadence Virtuoso  \\
            \textbf{Office Applications:} \LaTeX 
\end{spacing}
                                    
} %% technical skills done
    %% Teaching ---------------------------------      
    \header{TEACHING}
{
    \section{\centerline{Feb.2017-Jun.2017}}
    Teaching Assistant of \emph{Introduction to Information Science and Technology} 
     \vspace{-3mm}  
     \section{\centerline{Sept.2017-Jan.2018}}
     Teaching Assistant of \emph{Electric Circuits} 
} % teaching done
  \end{resume}

\end{document}        